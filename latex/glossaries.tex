\newterm{disciplina}{DCC067 Computação Evolucionista}

\newacronym{dcc}{DCC}{Departamento de Ciência da Computação}
\newacronym{ufjf}{UFJF}{Universidade Federal de Juiz de Fora}
\newacronym[
    description={Algoritmo Genético, do inglês \textit{Genetic Algorithm}, é uma técnica de otimização baseada em evolução biológica},
    plural={GAs},
    longplural={Algoritmos Genéticos}
]{ga}{GA}{Algoritmo Genético}
\newacronym[
    description={Congresso de Computação Evolucionária, do inglês \textit{Congress on Evolutionary Computation}, organizado pelo IEEE no ano de 2014 \cite{cec2014}},
]{cec14}{CEC 2014}{Congress on Evolutionary Computation 2014}

\newglossaryentry{crossover}{
    name={Crossover},
    text={\textit{crossover}},
    description={Operador genético que combina dois indivíduos para gerar um ou mais descendentes}
}
\newglossaryentry{opfunu}{
    name={Opfunu},
    text={\textit{opfunu}},
    description={Pacote de funções de benchmark para otimização \cite{opfunu_software}}
}
\newglossaryentry{mealpy}{
    name={Mealpy},
    text={\textit{mealpy}},
    description={Biblioteca de algoritmos evolutivos \cite{mealpy_software}}
}
\newglossaryentry{esga}{
    name={EliteSingleGA},
    text={\textit{EliteSingleGA}},
    description={Classe de algoritmo evolutivo da biblioteca \gls{mealpy}, que aplica método de elitismo
        },
    see={mealpy}
}
