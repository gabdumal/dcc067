\newterm{disciplina}{DCC067 Computação Evolucionista}

\newacronym{dcc}{DCC}{Departamento de Ciência da Computação}
\newacronym{ufjf}{UFJF}{Universidade Federal de Juiz de Fora}
\newacronym[
    description={Algoritmo Genético, do inglês \textit{Genetic Algorithm}, é uma técnica de otimização baseada em evolução biológica},
    plural={GAs},
    longplural={Algoritmos Genéticos}
]{ga}{GA}{Algoritmo Genético}
\newacronym[
    description={Congresso de Computação Evolucionária, do inglês \textit{Congress on Evolutionary Computation}, organizado pelo IEEE no ano de 2014 \cite{cec2014}},
]{cec14}{CEC 2014}{Congress on Evolutionary Computation 2014}
\newacronym[
    description={\texttt{Rotated High Conditioned Elliptic Function}. Função de benchmark convexa com alta condição de elipticidade, disponível no pacote \gls{opfunu} \cite{opfunu_software}},
    text={\texttt{Rotated High Conditioned Elliptic Function}}
]{f1}{CEC F1 2014}{\texttt{Rotated High Conditioned Elliptic Function}
}
\newacronym[
    description={\texttt{Shifted and Rotated Ackley's Function}. Função de benchmark não convexa de Ackley com deslocamento e rotação, disponível no pacote \gls{opfunu} \cite{opfunu_software}},
    text={\texttt{Shifted and Rotated Ackley's Function}}
]{f5}{CEC F5 2014}{\texttt{Shifted and Rotated Ackley's Function}}

\newglossaryentry{crossover}{
    name={Crossover},
    text={\textit{crossover}},
    description={Operador genético que combina dois indivíduos para gerar um ou mais descendentes}
}
\newglossaryentry{opfunu}{
    name={Opfunu},
    text={\textit{opfunu}},
    description={Pacote de funções de benchmark para otimização \cite{opfunu_software}}
}
\newglossaryentry{mealpy}{
    name={Mealpy},
    text={\textit{mealpy}},
    description={Biblioteca de algoritmos evolutivos \cite{mealpy_software}}
}
\newglossaryentry{esga}{
    name={EliteSingleGA},
    text={\textit{EliteSingleGA}},
    description={Classe de algoritmo evolutivo da biblioteca \gls{mealpy}, que aplica método genético com elitismo},
    see={mealpy}
}
\newglossaryentry{oma}{
    name={OriginalMA},
    text={\textit{OriginalMA}},
    description={Classe de algoritmo evolutivo da biblioteca \gls{mealpy}, que aplica método memético},
    see={mealpy}
}
\newglossaryentry{fitness}{
    name={Fitness},
    text={\textit{fitness}},
    description={Medida de qualidade de um indivíduo em um algoritmo evolutivo}
}
\newglossaryentry{problem}{
    name={Problema da Treliça de Três Barras},
    text={Problema da Treliça de Três Barras},
    description={Problema de otimização com restrições que envolve a minimização do peso de uma treliça de três barras}
}