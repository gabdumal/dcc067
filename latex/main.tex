\documentclass[
    % -- opções da classe memoir --
    article,			% indica que é um artigo acadêmico
    11pt,				% tamanho da fonte
    oneside,			% para impressão apenas no recto. Oposto a twoside
    a4paper,			% tamanho do papel. 
    % -- opções da classe abntex2 --
    %chapter=TITLE,		% títulos de capítulos convertidos em letras maiúsculas
    %section=TITLE,		% títulos de seções convertidos em letras maiúsculas
    %subsection=TITLE,	% títulos de subseções convertidos em letras maiúsculas
    %subsubsection=TITLE % títulos de subsubseções convertidos em letras maiúsculas
    % -- opções do pacote babel --
    english,			% idioma adicional para hifenização
    brazil,				% o último idioma é o principal do documento
    sumario=tradicional
]{abntex2}

% -----
% Pacotes fundamentais 
% -----
\usepackage{lmodern}			% Usa a fonte Latin Modern
\usepackage[T1]{fontenc}		% Seleção de códigos de fonte.
\usepackage[utf8]{inputenc}		% Codificação do documento (conversão automática dos acentos)
\usepackage{indentfirst}		% Identa o primeiro parágrafo de cada seção.
\usepackage{nomencl} 			% Lista de símbolos
\usepackage{color}				% Controle das cores
\usepackage{graphicx}			% Inclusão de gráficos
\usepackage{microtype} 			% Para melhorias de justificação
\usepackage[xindy={language=portuguese},subentrycounter,seeautonumberlist,nonumberlist=true]{glossaries}
\usepackage[brazilian,hyperpageref]{backref}	 % Páginas com as citações na bibliografia
\usepackage[alf]{abntex2cite}	% Citações padrão ABNT
% -----

% -----
% Configurações de pacotes
% -----

% --- Glossaries ---
\newterm{disciplina}{DCC067 Computação Evolucionista}

\newacronym{dcc}{DCC}{Departamento de Ciência da Computação}
\newacronym{ufjf}{UFJF}{Universidade Federal de Juiz de Fora}
\newacronym[
    description={Algoritmo Genético, do inglês \textit{Genetic Algorithm}, é uma técnica de otimização baseada em evolução biológica},
    plural={GAs},
    longplural={Algoritmos Genéticos}
]{ga}{GA}{Algoritmo Genético}
\newacronym[
    description={Congresso de Computação Evolucionária, do inglês \textit{Congress on Evolutionary Computation}, organizado pelo IEEE no ano de 2014 \cite{cec2014}},
]{cec14}{CEC 2014}{Congress on Evolutionary Computation 2014}
\newacronym[
    description={\texttt{Rotated High Conditioned Elliptic Function}. Função de benchmark convexa com alta condição de elipticidade, disponível no pacote \gls{opfunu} \cite{opfunu_software}},
    text={\texttt{Rotated High Conditioned Elliptic Function}}
]{f1}{CEC F1 2014}{\texttt{Rotated High Conditioned Elliptic Function}
}
\newacronym[
    description={\texttt{Shifted and Rotated Ackley's Function}. Função de benchmark não convexa de Ackley com deslocamento e rotação, disponível no pacote \gls{opfunu} \cite{opfunu_software}},
    text={\texttt{Shifted and Rotated Ackley's Function}}
]{f5}{CEC F5 2014}{\texttt{Shifted and Rotated Ackley's Function}}

\newglossaryentry{crossover}{
    name={Crossover},
    text={\textit{crossover}},
    description={Operador genético que combina dois indivíduos para gerar um ou mais descendentes}
}
\newglossaryentry{opfunu}{
    name={Opfunu},
    text={\textit{opfunu}},
    description={Pacote de funções de benchmark para otimização \cite{opfunu_software}}
}
\newglossaryentry{mealpy}{
    name={Mealpy},
    text={\textit{mealpy}},
    description={Biblioteca de algoritmos evolutivos \cite{mealpy_software}}
}
\newglossaryentry{esga}{
    name={EliteSingleGA},
    text={\textit{EliteSingleGA}},
    description={Classe de algoritmo evolutivo da biblioteca \gls{mealpy}, que aplica método genético com elitismo},
    see={mealpy}
}
\newglossaryentry{oma}{
    name={OriginalMA},
    text={\textit{OriginalMA}},
    description={Classe de algoritmo evolutivo da biblioteca \gls{mealpy}, que aplica método memético},
    see={mealpy}
}
\newglossaryentry{fitness}{
    name={Fitness},
    text={\textit{fitness}},
    description={Medida de qualidade de um indivíduo em um algoritmo evolutivo}
}
\newglossaryentry{problem}{
    name={Problema da Treliça de Três Barras},
    text={Problema da Treliça de Três Barras},
    description={Problema de otimização com restrições que envolve a minimização do peso de uma treliça de três barras}
}

% --- Backref ---
% Texto padrão antes do número das páginas
\renewcommand{\backref}{}
% Define os textos da citação
\renewcommand*{\backrefalt}[4]{
	\ifcase{} #1 %
		Nenhuma citação no texto.%
	\or{}
		Citado na página #2.%
	\else
		Citado #1 vezes nas páginas #2.%
	\fi
}

% -----

% -----
% Informações de dados para CAPA e FOLHA DE ROSTO
% -----
\titulo{DCC067 Computação Evolucionista: Trabalho Prático 1 --- Avaliação do uso de seleção por roleta e por torneio com variações de K, e avaliação do uso de crossover de um ponto e aritmético}
\tituloestrangeiro{DCC067 Evolutionary Computing: Practical Work 1 --- Evaluation of the use of roulette and tournament selection with variations of K, and evaluation of the use of one-point and arithmetic crossover}

\autor{%
    Celso Gabriel Malosto \and Lucas Paiva Santos \and Victor Duque Pinto
}

\local{Juiz de Fora}
\data{2024}
% -----

% -----
% Configurações de aparência do PDF final
% -----
\definecolor{blue}{RGB}{41,5,195}
% Informações do PDF
\makeatletter
\hypersetup{%
     	%pagebackref=true,
		pdftitle={\@title}, 
		pdfauthor={\@author},
    	pdfsubject={Modelo de artigo científico com abnTeX2},
	    pdfcreator={LaTeX with abnTeX2},
		pdfkeywords={abnt}{latex}{abntex}{abntex2}{artigo científico}, 
		colorlinks=true,       		% false: boxed links; true: colored links
    	linkcolor=blue,          	% color of internal links
    	citecolor=blue,        		% color of links to bibliography
    	filecolor=magenta,      		% color of file links
		urlcolor=blue,
		bookmarksdepth=4
}
\makeatother
% -----

% -----
% Demais configurações
% -----

% Compila o índice
\makeindex

% Altera as margens padrões
\setlrmarginsandblock{3cm}{3cm}{*}
\setulmarginsandblock{3cm}{3cm}{*}
\checkandfixthelayout{}

% Espaçamentos entre linhas e parágrafos
\setlength{\parindent}{1.3cm} % Tamanho do parágrafo
\setlength{\parskip}{0.2cm}  % Controle do espaçamento entre um parágrafo e outro
\SingleSpacing{} % Espaçamento simples

% -----

\begin{document}

% -----
% Configurações do documento
% -----
% Seleciona o idioma do documento
\selectlanguage{brazil}

% Retira espaço extra obsoleto entre as frases
\frenchspacing{}
% -----

% =====
% ELEMENTOS PRÉ-TEXTUAIS
% =====
\pretextual{}

% Página de titulo principal (obrigatório)
\maketitle{}

% Resumos
% Resumo em português
\begin{resumoumacoluna}
    Este relatório tem como intuito analisar o comportamento de algoritmo genéticos de acordo com operadores específicos. Assim, foi utilizada a biblioteca MEALPY, implementada na linguagem Python que possui diversos algoritmos meta-heurísticos, incluindo o genético. A fim de fazer uma comparação e apresentar resultados, foram avaliados diferentes métodos de seleção (roleta e torneio) e de \textit{crossover} (um ponto e aritmético), variando os valores associados, como as probabilidades. Além disso, para realização dos testes, utilizou-se duas funções de avaliação da biblioteca opfunu, uma convexa e outra não-convexa, as quais foram propriamente definidas nos trabalhos de referência. Ao final, foi feita uma discussão e conclusão sobre os resultados obtidos nos testes.

    \noindent
    \textbf{Palavras-chave}: computação evolucionista. algoritmo genético. método de seleção. \textit{crossover}. função de avaliação.
\end{resumoumacoluna}

% Resumo em inglês
\renewcommand{\resumoname}{Abstract}
\begin{resumoumacoluna}
    \begin{otherlanguage*}{english}

        This report aims to analyze the behavior of genetic algorithms according to specific operators. It uses the MEALPY library, which is implemented in Python and contains various meta-heuristic algorithms, including genetic algorithms. In order to compare and present results, different selection methods (roulette and tournament) and crossover methods (one-point and arithmetic) were evaluated, varying the associated values, such as probabilities. In addition, two evaluation functions from the opfunu library were used to carry out the tests, one convex and the other non-convex, which were properly defined in the reference works. At the end, a discussion and conclusion was drawn on the results obtained in the tests.

        \vspace{\onelineskip}

        \noindent
        \textbf{Keywords}: evolutionary computing. genetic algorithm. selection method. crossover. evaluation function.
    \end{otherlanguage*}
\end{resumoumacoluna}


\begin{center}\smaller{}
    \textbf{Data de submissão e aprovação}: 28 de agosto de 2024.
\end{center}

% =====

% =====
% ELEMENTOS TEXTUAIS
% =====
\textual{}

Hello, World! \gls{pai}

% =====

% =====
% ELEMENTOS PÓS-TEXTUAIS
% =====
\postextual{}

% --- Referências ---
\bibliography{bibliography}

% --- Glossário ---
\printglossaries{}

% =====

\end{document}