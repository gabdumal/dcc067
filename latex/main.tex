\documentclass[
    % -- opções da classe memoir --
    article,			% indica que é um artigo acadêmico
    11pt,				% tamanho da fonte
    oneside,			% para impressão apenas no recto. Oposto a twoside
    a4paper,			% tamanho do papel. 
    % -- opções da classe abntex2 --
    %chapter=TITLE,		% títulos de capítulos convertidos em letras maiúsculas
    %section=TITLE,		% títulos de seções convertidos em letras maiúsculas
    %subsection=TITLE,	% títulos de subseções convertidos em letras maiúsculas
    %subsubsection=TITLE % títulos de subsubseções convertidos em letras maiúsculas
    % -- opções do pacote babel --
    english,			% idioma adicional para hifenização
    brazil,				% o último idioma é o principal do documento
    sumario=tradicional
]{abntex2}

% -----
% Pacotes fundamentais 
% -----
\usepackage{lmodern}			% Usa a fonte Latin Modern
\usepackage[T1]{fontenc}		% Seleção de códigos de fonte.
\usepackage[utf8]{inputenc}		% Codificação do documento (conversão automática dos acentos)
\usepackage{indentfirst}		% Identa o primeiro parágrafo de cada seção.
\usepackage{nomencl} 			% Lista de símbolos
\usepackage{color}				% Controle das cores
\usepackage{graphicx}			% Inclusão de gráficos
\usepackage{microtype} 			% Para melhorias de justificação
\usepackage[xindy={language=portuguese},subentrycounter,seeautonumberlist,nonumberlist=true]{glossaries}
\usepackage[brazilian,hyperpageref]{backref}	 % Páginas com as citações na bibliografia
\usepackage[alf]{abntex2cite}	% Citações padrão ABNT
% -----

% -----
% Configurações de pacotes
% -----

% --- Glossaries ---
\newglossaryentry{disciplina}{%
    name={DCC067 Computação Evolucionista},
    description={este é uma entrada pai, que possui outras subentradas
        }
}

\newacronym{dcc}{DCC}{Departamento de Ciência da Computação}


% --- Backref ---
% Texto padrão antes do número das páginas
\renewcommand{\backref}{}
% Define os textos da citação
\renewcommand*{\backrefalt}[4]{
	\ifcase{} #1 %
		Nenhuma citação no texto.%
	\or{}
		Citado na página #2.%
	\else
		Citado #1 vezes nas páginas #2.%
	\fi
}

% -----

% -----
% Informações de dados para CAPA e FOLHA DE ROSTO
% -----
\titulo{DCC067 Computação Evolucionista: Trabalho Prático 1 --- Avaliação do uso de seleção por roleta e por torneio com variações de K, e avaliação do uso de crossover de um ponto e aritmético}
\tituloestrangeiro{DCC067 Evolutionary Computing: Practical Work 1 --- Evaluation of the use of roulette and tournament selection with variations of K, and evaluation of the use of one-point and arithmetic crossover}

\autor{%
    Celso Gabriel Malosto \and Lucas Paiva Santos \and Victor Duque Pinto
}

\local{Juiz de Fora}
\data{2024}
% -----

% -----
% Configurações de aparência do PDF final
% -----
\definecolor{blue}{RGB}{41,5,195}
% Informações do PDF
\makeatletter
\hypersetup{%
     	%pagebackref=true,
		pdftitle={\@title}, 
		pdfauthor={\@author},
    	pdfsubject={Modelo de artigo científico com abnTeX2},
	    pdfcreator={LaTeX with abnTeX2},
		pdfkeywords={abnt}{latex}{abntex}{abntex2}{artigo científico}, 
		colorlinks=true,       		% false: boxed links; true: colored links
    	linkcolor=blue,          	% color of internal links
    	citecolor=blue,        		% color of links to bibliography
    	filecolor=magenta,      		% color of file links
		urlcolor=blue,
		bookmarksdepth=4
}
\makeatother
% -----

% -----
% Demais configurações
% -----

% Compila o índice
\makeindex

% Altera as margens padrões
\setlrmarginsandblock{3cm}{3cm}{*}
\setulmarginsandblock{3cm}{3cm}{*}
\checkandfixthelayout{}

% Espaçamentos entre linhas e parágrafos
\setlength{\parindent}{1.3cm} % Tamanho do parágrafo
\setlength{\parskip}{0.2cm}  % Controle do espaçamento entre um parágrafo e outro
\SingleSpacing{} % Espaçamento simples

% -----

\begin{document}

% -----
% Configurações do documento
% -----
% Seleciona o idioma do documento
\selectlanguage{brazil}

% Retira espaço extra obsoleto entre as frases
\frenchspacing{}
% -----

% =====
% ELEMENTOS PRÉ-TEXTUAIS
% =====
\pretextual{}

% Página de titulo principal (obrigatório)
\maketitle{}

% Resumos
% Resumo em português
\begin{resumoumacoluna}
    Conforme a ABNT NBR 6022:2018, o resumo no idioma do documento é elemento obrigatório.
    Constituído de uma sequência de frases concisas e objetivas e não de uma
    simples enumeração de tópicos, não ultrapassando 250 palavras, seguido, logo
    abaixo, das palavras representativas do conteúdo do trabalho, isto é,
    palavras-chave e/ou descritores, conforme a NBR 6028. (\ldots) As
    palavras-chave devem figurar logo abaixo do resumo, antecedidas da expressão
    Palavras-chave:, separadas entre si por ponto e finalizadas também por ponto.

    \vspace{\onelineskip}

    \noindent
    \textbf{Palavras-chave}: latex. abntex. editoração de texto.
\end{resumoumacoluna}

% Resumo em inglês
\renewcommand{\resumoname}{Abstract}
\begin{resumoumacoluna}
    \begin{otherlanguage*}{english}
        According to ABNT NBR 6022:2018, an abstract in foreign language is optional.

        \vspace{\onelineskip}

        \noindent
        \textbf{Keywords}: latex. abntex.
    \end{otherlanguage*}
\end{resumoumacoluna}


\begin{center}\smaller{}
    \textbf{Data de submissão e aprovação}: 28 de agosto de 2024.
\end{center}

% =====

% =====
% ELEMENTOS TEXTUAIS
% =====
\textual{}

Hello, World! \gls{pai}

% =====

% =====
% ELEMENTOS PÓS-TEXTUAIS
% =====
\postextual{}

% --- Referências ---
\bibliography{bibliography}

% --- Glossário ---
\printglossaries{}

% =====

\end{document}