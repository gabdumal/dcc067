\section{Conclusão}%
\label{sec:conclusao}

Os resultados dos experimentos confirmaram as expectativas em relação aos pesos de penalização baixos, esses valores levaram ambos os algoritmos a buscar o mínimo da função objetivo de forma indiscriminada, violando as restrições do problema. Essa falta de penalização adequada fez com que as soluções fossem estruturalmente inviáveis.

Por outro lado, pesos elevados de penalização, embora tenham impactado o tempo de convergência, não impediram que os algoritmos encontrassem bons resultados, com isso observa-se à baixa complexidade do problema em questão, que envolve uma função simples com apenas duas variáveis, sendo o provável motivo deste comportamento inesperado já que penalização não foi capaz de impedir a eficiência dos algoritmos em identificar soluções próximas ao ótimo.

Um ponto notável foi o desempenho do resolvedor local do algoritmo memético. Em várias execuções, o algoritmo convergiu em uma única iteração, mostrando o componente local suficiente para alcançar resultados satisfatórios, novamente, a simplicidade do problema parece ter favorecido o desempenho, tornando o comportamento evolutivo do algoritmo pouco expressivo.