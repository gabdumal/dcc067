% Resumo em português
\begin{resumoumacoluna}
    Este relatório tem como intuito analisar o comportamento de algoritmo genéticos de acordo com operadores específicos. Assim, foi utilizada a biblioteca MEALPY, implementada na linguagem Python que possui diversos algoritmos meta-heurísticos, incluindo o genético. A fim de fazer uma comparação e apresentar resultados, foram avaliados diferentes métodos de seleção (roleta e torneio) e de \textit{crossover} (um ponto e aritmético), variando os valores associados, como as probabilidades. Além disso, para realização dos testes, utilizou-se duas funções de avaliação da biblioteca opfunu, uma convexa e outra não-convexa, as quais foram propriamente definidas nos trabalhos de referência. Ao final, foi feita uma discussão e conclusão sobre os resultados obtidos nos testes.

    \noindent
    \textbf{Palavras-chave}: computação evolucionista. algoritmo genético. método de seleção. \textit{crossover}. função de avaliação.
\end{resumoumacoluna}

% Resumo em inglês
\renewcommand{\resumoname}{Abstract}
\begin{resumoumacoluna}
    \begin{otherlanguage*}{english}

        This report aims to analyze the behavior of genetic algorithms according to specific operators. It uses the MEALPY library, which is implemented in Python and contains various meta-heuristic algorithms, including genetic algorithms. In order to compare and present results, different selection methods (roulette and tournament) and crossover methods (one-point and arithmetic) were evaluated, varying the associated values, such as probabilities. In addition, two evaluation functions from the opfunu library were used to carry out the tests, one convex and the other non-convex, which were properly defined in the reference works. At the end, a discussion and conclusion was drawn on the results obtained in the tests.

        \vspace{\onelineskip}

        \noindent
        \textbf{Keywords}: evolutionary computing. genetic algorithm. selection method. crossover. evaluation function.
    \end{otherlanguage*}
\end{resumoumacoluna}
