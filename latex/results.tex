\section{Resultados}%
\label{sec:resultados}

Nesta seção, são apresentados os resultados obtidos a partir da execução dos experimentos descritos na \autoref{sec:metodologia}.
Os valores obtidos pelo experimento são apresentados em planilhas eletrônicas, disponíveis para os problemas em 10 dimensões\footnote{Resultados do experimento para problema de 10 dimensões. Disponível em: \url{https://docs.google.com/spreadsheets/d/1bOAUmGv3ZejnVMTojSYcHIgnF4cwjmHWzkgA4WuXQwg/edit?usp=sharing}} e em 20 dimensões\footnote{Resultados do experimento para problema de 20 dimensões. Disponível em: \url{https://docs.google.com/spreadsheets/d/1HRjwZU50QHC8F-N6ofHfJgo7oWcU0-PTe3e8cokZ4Gc/edit?usp=sharing}}.

\subsection{Análise dos Resultados}%

A partir dos resultados obtidos, é possível analisar o desempenho dos algoritmos em cada um dos problemas propostos.
Para isso, com base nos valores de \gls{fitness} das execuções, foram calculadas as médias, os desvios padrão --- representados por \( \sigma \) --- e a discrepância da solução encontrada em relação à solução ótima conhecida, além de listar o melhor valor dentre as 10 execuções de cada variante.
A \autoref{tab:fitness_10} apresenta os resultados obtidos para os problemas \gls{f1} e \gls{f5} em 10 dimensões. Já a \autoref{tab:fitness_20} apresenta os resultados obtidos para os problemas \gls{f1} e \gls{f5} em 20 dimensões.


\begin{table}[!ht]
    \resizebox{\textwidth}{!}{%
        \begin{tabular}{llllrrrr}
            \bottomrule
            \textbf{Obj.\ func.}                       &
            \textbf{Crossover}                         &
            \textbf{Selection}                         &
            \textbf{\( \boldsymbol{K} \)}              &
            \textbf{Best fitness}                      &
            \textbf{Average fitness}                   &
            \textbf{Fitness \( \boldsymbol{\sigma} \)} &
            \textbf{Discrepancy}                                                                                                                      \\ \midrule
            F12014                                     & one\_point & roulette   & None & 5,427,918.34 & 8,721,732.84  & 2,010,993.99 & 8,721,632.84  \\
            F12014                                     & one\_point & tournament & 0.1  & 1,984,810.71 & 5,039,684.36  & 2,413,869.42 & 5,039,584.36  \\
            F12014                                     & one\_point & tournament & 0.2  & 626,253.43   & 2,977,922.47  & 1,469,234.28 & 2,977,822.47  \\
            F12014                                     & one\_point & tournament & 0.3  & 251,279.86   & 1,967,444.84  & 976,822.48   & 1,967,344.84  \\
            F12014                                     & one\_point & tournament & 0.4  & 136,152.15   & 778,614.73    & 474,529.84   & 778,514.73    \\
            F12014                                     & one\_point & tournament & 0.5  & 125,213.30   & 300,636.40    & 134,755.61   & 300,536.40    \\
            F12014                                     & arithmetic & roulette   & None & 9,355,133.39 & 15,026,194.68 & 2,507,612.16 & 15,026,094.68 \\
            F12014                                     & arithmetic & tournament & 0.1  & 4,044,396.31 & 5,881,057.14  & 1,269,891.90 & 5,880,957.14  \\
            F12014                                     & arithmetic & tournament & 0.2  & 332,668.03   & 839,066.42    & 342,200.41   & 838,966.42    \\
            F12014                                     & arithmetic & tournament & 0.3  & 179,742.67   & 345,012.16    & 141,971.91   & 344,912.16    \\
            F12014                                     & arithmetic & tournament & 0.4  & 98,441.27    & 181,654.00    & 79,546.79    & 181,554.00    \\
            F12014                                     & arithmetic & tournament & 0.5  & 60,784.82    & 143,231.77    & 60,174.04    & 143,131.77    \\
            F52014                                     & one\_point & roulette   & None & 520.15       & 520.21        & 0.03         & 20.21         \\
            F52014                                     & one\_point & tournament & 0.1  & 519.22       & 520.10        & 0.31         & 20.10         \\
            F52014                                     & one\_point & tournament & 0.2  & 519.63       & 520.03        & 0.14         & 20.03         \\
            F52014                                     & one\_point & tournament & 0.3  & 510.25       & 518.30        & 3.39         & 18.30         \\
            F52014                                     & one\_point & tournament & 0.4  & 507.63       & 516.73        & 5.17         & 16.73         \\
            F52014                                     & one\_point & tournament & 0.5  & 519.48       & 519.83        & 0.22         & 19.83         \\
            F52014                                     & arithmetic & roulette   & None & 520.01       & 520.23        & 0.10         & 20.23         \\
            F52014                                     & arithmetic & tournament & 0.1  & 518.99       & 520.13        & 0.41         & 20.13         \\
            F52014                                     & arithmetic & tournament & 0.2  & 508.05       & 512.63        & 5.02         & 12.63         \\
            F52014                                     & arithmetic & tournament & 0.3  & 506.66       & 507.89        & 0.86         & 7.89          \\
            F52014                                     & arithmetic & tournament & 0.4  & 506.47       & 508.79        & 3.83         & 8.79          \\
            F52014                                     & arithmetic & tournament & 0.5  & 506.51       & 508.78        & 4.05         & 8.78          \\
            \toprule
        \end{tabular}%
    }
    \caption{Resultados obtidos para os problemas \glsentryname{f1} e \glsentryname{f5} em 10 dimensões.
    }%
    \label{tab:fitness_10}
\end{table}

\begin{table}[!ht]
    \resizebox{\textwidth}{!}{%
        \begin{tabular}{llllrrrr}
            \bottomrule
            \textbf{Obj.\ func.}                       &
            \textbf{Crossover}                         &
            \textbf{Selection}                         &
            \textbf{\( \boldsymbol{K} \)}              &
            \textbf{Best fitness}                      &
            \textbf{Average fitness}                   &
            \textbf{Fitness \( \boldsymbol{\sigma} \)} &
            \textbf{Discrepancy}                                                                                                                           \\ \midrule
            F12014                                     & one\_point & roulette   & None & 31,333,596.32  & 51,840,884.76  & 13,297,083.01 & 51,840,784.76  \\
            F12014                                     & one\_point & tournament & 0.1  & 6,459,590.39   & 8,170,932.28   & 1,340,563.86  & 8,170,832.28   \\
            F12014                                     & one\_point & tournament & 0.2  & 501,184.77     & 1,011,075.90   & 513,203.52    & 1,010,975.90   \\
            F12014                                     & one\_point & tournament & 0.3  & 216,342.15     & 703,931.58     & 525,414.34    & 703,831.58     \\
            F12014                                     & one\_point & tournament & 0.4  & 213,078.25     & 699,594.39     & 317,577.00    & 699,494.39     \\
            F12014                                     & one\_point & tournament & 0.5  & 257,286.41     & 850,689.20     & 462,953.51    & 850,589.20     \\
            F12014                                     & arithmetic & roulette   & None & 149,508,377.65 & 194,305,830.34 & 31,394,179.76 & 194,305,730.34 \\
            F12014                                     & arithmetic & tournament & 0.1  & 4,554,550.01   & 6,522,961.69   & 1,494,523.52  & 6,522,861.69   \\
            F12014                                     & arithmetic & tournament & 0.2  & 418,897.70     & 658,253.99     & 124,383.49    & 658,153.99     \\
            F12014                                     & arithmetic & tournament & 0.3  & 202,876.41     & 402,340.97     & 176,709.61    & 402,240.97     \\
            F12014                                     & arithmetic & tournament & 0.4  & 143,574.03     & 506,514.55     & 323,014.25    & 506,414.55     \\
            F12014                                     & arithmetic & tournament & 0.5  & 124,955.96     & 409,121.44     & 215,047.98    & 409,021.44     \\
            F52014                                     & one\_point & roulette   & None & 520.19         & 520.24         & 0.03          & 20.24          \\
            F52014                                     & one\_point & tournament & 0.1  & 520.16         & 520.18         & 0.02          & 20.18          \\
            F52014                                     & one\_point & tournament & 0.2  & 520.04         & 520.07         & 0.02          & 20.07          \\
            F52014                                     & one\_point & tournament & 0.3  & 520.06         & 520.06         & 0.01          & 20.06          \\
            F52014                                     & one\_point & tournament & 0.4  & 520.04         & 520.05         & 0.01          & 20.05          \\
            F52014                                     & one\_point & tournament & 0.5  & 520.04         & 520.05         & 0.01          & 20.05          \\
            F52014                                     & arithmetic & roulette   & None & 520.60         & 520.70         & 0.07          & 20.70          \\
            F52014                                     & arithmetic & tournament & 0.1  & 520.53         & 520.70         & 0.07          & 20.70          \\
            F52014                                     & arithmetic & tournament & 0.2  & 509.85         & 518.68         & 4.02          & 18.68          \\
            F52014                                     & arithmetic & tournament & 0.3  & 509.64         & 517.27         & 5.14          & 17.27          \\
            F52014                                     & arithmetic & tournament & 0.4  & 520.30         & 520.38         & 0.03          & 20.38          \\
            F52014                                     & arithmetic & tournament & 0.5  & 520.26         & 520.31         & 0.03          & 20.31          \\
            \toprule
        \end{tabular}%
    }
    \caption{Resultados obtidos para os problemas \glsentryname{f1} e \glsentryname{f5} em 20 dimensões.
    }%
    \label{tab:fitness_20}
\end{table}

Nos casos em que o operador de seleção por torneio foi utilizado para a função \gls{f1}, o valor de \( K \) foi modificado entre 0,1 e 0,5 aumentando sempre em 0,1.
Neste caso foi possível observar uma ligeira piora da solução em ambos os extremos.
O valor que encontrou os melhores resultados foi 0,2 (padrão da biblioteca).
Foi bastante notório que este valor levava o algoritmo a convergir mais tarde, com cerca de 4000 gerações.
Os valores associados ao elitismo não foram alterados, mantendo-se os padrões da biblioteca, com 0,1 para o elitismo de melhores indivíduos e 0,3 para o de piores.

Acerca das funções de benchmark, a \gls{f1} apresentou resultados menos satisfatórios, com discrepâncias de até 15.026.094,68 e 194.305.730,34 em 10 e 20 dimensões, respectivamente.
Estes valores são bastante elevados, considerando que a função \gls{f1} é convexa e possui um mínimo global conhecido de 100.

Por outro lado, a função \gls{f5} apresentou resultados mais satisfatórios, com discrepâncias de até 20,23 e 20,70 em 10 e 20 dimensões, respectivamente, levando em consideração que a função \gls{f5} é não convexa e possui um mínimo global conhecido de 500.
Estes valores são mais próximos do esperado, indicando que o \gls{ga} apresenta características que o tornam mais eficiente para este tipo de função.

A seleção por roleta esteve entre os piores resultados. Seus valores de \gls{fitness} médio figuravam como 15.026.194,68 e 194.305.830,34, para 10 e 20 dimensões.
A seleção por torneio, por outro lado, apresentou resultados mais satisfatórios.
Considerando os valores de \gls{fitness} médios, a seleção por torneio com \( K \) em 0.3 foi a que obteve os melhores resultados para a função \gls{f1} em 20 dimensões, com 402,340.97.
Já para o experimento com 10 dimensões, houve uma melhora contínua conforme se aumentava o \( K \), até 0.5, com 143.231,77.
É possível propor a hipótese de que uma menor pressão seletiva, como a proporcionada por \( K \) em valores maiores, pode levar a uma maior diversidade genética, o que pode ser benéfico para a convergência do algoritmo.

Apesar de o operador de \gls{crossover} por aritmética ter apresentado o pior resultado para a função \gls{f1}, com 15.026.194,68 para 10 dimensões, deve-se observar que, neste caso, ele estava associado à seleção por roleta.
Quando o observa junto à seleção por torneio com \( K \) em 0.5, o valor de adequação médio obtido foi de 143.231,77, o melhor para essa função nesse número de dimensões.

A função \gls{f5}, por apresentar menos diferença entre os valores de adequação das soluções das variantes, tornou mais difícil a identificação de padrões.
Ainda assim, pôde-se perceber que em 10 dimensões, com o torneio exibindo \( K \) em 0.4 e 0.3 associado ao operador de \gls{crossover} um ponto teve vantagem em relação aos outros valores de \( k \) e roleta.
De todo modo, o operador de \gls{crossover} aritmético obtém vantagem significativa com qualquer valor de \( K \) maior que 0.1.

Quanto à análise com 20 dimensões, a diferença entre as variantes foi ainda menos acentuada.
Apenas se constatou que o operador de \gls{crossover} aritmético gerou resultados ligeiramente superiores quando o torneio foi utilizado com \( K \) em 0.2 e 0.3.