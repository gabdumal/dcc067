\section{Resultados}%
\label{sec:resultados}

Nesta seção, são apresentados os resultados obtidos a partir da execução dos experimentos descritos na \autoref{sec:metodologia}.
% Os valores obtidos pelo experimento são apresentados em planilhas eletrônicas, disponíveis para os problemas em 10 dimensões\footnote{Resultados do experimento para problema de 10 dimensões. Disponível em: \url{https://docs.google.com/spreadsheets/d/1bOAUmGv3ZejnVMTojSYcHIgnF4cwjmHWzkgA4WuXQwg/edit?usp=sharing}} e em 20 dimensões\footnote{Resultados do experimento para problema de 20 dimensões. Disponível em: \url{https://docs.google.com/spreadsheets/d/1HRjwZU50QHC8F-N6ofHfJgo7oWcU0-PTe3e8cokZ4Gc/edit?usp=sharing}}.

\subsection{Análise dos Resultados}%

A partir dos resultados obtidos, é possível analisar o desempenho dos algoritmos em cada um dos problemas propostos.
Foi considerado o valor de \gls{fitness} do ótimo global ideal como 263.895843.

Com base nos valores de \gls{fitness} das execuções, foram calculados:

\begin{symbols}
    \item[\( \Delta \)] a discrepância em relação ao ótimo global ideal;
    \item[\( \sigma^{2} \)] a variância dos valores de \gls{fitness} obtidos;
    \item[\( \sigma \)] o desvio padrão dos valores de \gls{fitness} obtidos.
\end{symbols}

O experimento consistiu em executar cada algoritmo 10 vezes, variando os pesos atribuídos às penalizações por violação das restrições. Inicialmente, testaram-se pesos de 100, 200, 400 e 800, com o valor dobrando a cada execução. Posteriormente, o peso foi reduzido gradualmente, sendo testados 50, 25, 12,5 e, por fim, 0.

As tabelas~\ref{tab:resultados-genetico} e~\ref{tab:resultados-memetico} apresentam, para os otimizadores \gls{esga} e \gls{oma}, respectivamente, os resultados obtidos.
Elencam-se os valores de \gls{fitness} e de discrepância da melhor solução e da média das 10 execuções realizadas para cada peso de penalização, além de os valores de variância e desvio padrão entre as soluções encontradas.
Os dados completos do experimento com o algoritmo genético%
\footnote{%
    Resultados do experimento para o algoritmo genético. Disponível em: \url{https://docs.google.com/spreadsheets/d/1589Xh2bLunegQypvfnBg_3csuZElcqJtsFiOOzEYmz8/edit?usp=sharing}.
}
e com o memético%
\footnote{%
    Resultados do experimento para o algoritmo memético. Disponível em: \url{https://docs.google.com/spreadsheets/d/1d157cRAJqgtAQ1KfIHSeW07LFzINioJR8bucLpOU43w/edit?usp=drive_link}.
}
estão disponíveis on-line.

\begin{table}[!ht]
    \resizebox{\textwidth}{!}{%
        \begin{tabular}{lrrrrrrr}
            \bottomrule
            \textbf{Weight}                                &
            \textbf{Best fitness}                          &
            \textbf{Average fitness}                       &
            \textbf{Best \(\boldsymbol{\Delta}\)}          &
            \textbf{Average \(\boldsymbol{\Delta}\)}       &
            \textbf{Fitness \( \boldsymbol{\sigma^{2}} \)} &
            \textbf{Fitness \( \boldsymbol{\sigma}\)}
            \\ \midrule
            0                                              &
            \textbf{0.664898658369470}                     &
            \textbf{1.351379134170120}                     &
            261.310316805323000                            &
            262.544463865830000                            &
            0.436608119613905                              &
            0.660763285612862
            \\
            12.5                                           &
            178.997677252294000                            &
            179.104808940107000                            &
            84.504691568419000                             &
            84.791034059893200                             &
            0.013390962925215                              &
            0.115719328226597
            \\
            25                                             &
            216.367732200399000                            &
            216.393958331948000                            &
            47.428849151152000                             &
            47.501884668051500                             &
            0.000845483185771                              &
            0.029077193567664
            \\
            50                                             &
            263.916813724328000                            &
            263.963887551704000                            &
            0.020970724327981                              &
            0.068044551704293                              &
            0.000929526320056                              &
            0.030488134086159
            \\
            100                                            &
            263.907667139552000                            &
            263.929150603033000                            &
            0.011824139552005                              &
            \textbf{0.033307603033092}                     &
            0.000268368246443                              &
            0.016381948798706
            \\
            200                                            &
            263.897546338518000                            &
            263.940946098212000                            &
            \textbf{0.001703338517984}                     &
            0.045103098212383                              &
            0.000428792341542                              &
            0.020707301648019
            \\
            400                                            &
            263.904722244638000                            &
            263.941341028093000                            &
            0.008879244637967                              &
            0.045498028093277                              &
            0.000678108343263                              &
            0.026040513498460
            \\
            800                                            &
            263.901900217279000                            &
            263.931432135312000                            &
            0.006057217278965                              &
            0.035589135312284                              &
            \textbf{0.000207542501157}                     &
            \textbf{0.014406335452058}
            \\ \toprule
        \end{tabular}%
    }
    \caption{Resultados obtidos para o \gls{problem} em que se varia o peso da função de penalização, tendo sido aplicado o otimizador \gls{esga}.
    }%
    \label{tab:resultados-genetico}
\end{table}
\begin{table}[!ht]
    \resizebox{\textwidth}{!}{%
        \begin{tabular}{lrrrrrrr}
            \bottomrule
            \textbf{Weight}                                &
            \textbf{Best fitness}                          &
            \textbf{Average fitness}                       &
            \textbf{Best \(\boldsymbol{\Delta}\)}          &
            \textbf{Average \(\boldsymbol{\Delta}\)}       &
            \textbf{Fitness \( \boldsymbol{\sigma^{2}} \)} &
            \textbf{Fitness \( \boldsymbol{\sigma}\)}
            \\ \midrule
            0                                              &
            \textbf{0.000000000000000}                     &
            \textbf{0.000000000000000}                     &
            263.895843000000000                            &
            263.895843000000000                            &
            \textbf{0.000000000000000}                     &
            \textbf{0.000000000000000}
            \\
            12.5                                           &
            75.000000000000000                             &
            75.000000000000000                             &
            188.895843000000000                            &
            188.895843000000000                            &
            0.000000000000000                              &
            0.000000000000000
            \\
            25                                             &
            150.000000000000000                            &
            150.000000000000000                            &
            113.895843000000000                            &
            113.895843000000000                            &
            0.000000000000000                              &
            0.000000000000000
            \\
            50                                             &
            266.274169979695000                            &
            266.274169979695000                            &
            2.378326979694980                              &
            2.378326979694980                              &
            0.000000000000000                              &
            0.000000000000000
            \\
            100                                            &
            266.274169979695000                            &
            266.274169979695000                            &
            2.378326979694980                              &
            2.378326979694980                              &
            0.000000000000000                              &
            0.000000000000000
            \\
            200                                            &
            266.274169979695000                            &
            266.274169979695000                            &
            2.378326979694980                              &
            2.378326979694980                              &
            0.000000000000000                              &
            0.000000000000000
            \\
            400                                            &
            265.600129207775000                            &
            266.206765902503000                            &
            \textbf{1.704286207775000}                     &
            \textbf{2.310922902502980}                     &
            0.045433096221042                              &
            0.213150407508505
            \\
            800                                            &
            266.274169979695000                            &
            266.274169979695000                            &
            2.378326979694980                              &
            2.378326979694980                              &
            0.000000000000000                              &
            0.000000000000000
            \\ \toprule
        \end{tabular}%
    }
    \caption{Resultados obtidos para o \gls{problem} em que se varia o peso da função de penalização, tendo sido aplicado o otimizador \gls{oma}.
    }%
    \label{tab:resultados-memetico}
\end{table}

Em todas as execuções com pesos iguais ou superiores a 100, o algoritmo genético simples convergiu para soluções muito próximas do ótimo.
Mesmo com penalizações elevadas, como 800, a convergência se manteve consistente.
Entretanto, com esses pesos mais altos, o algoritmo exigiu um número maior de iterações para alcançar a solução final, embora a qualidade dessa tenha permanecido satisfatória.

\todo{colocar o conteúdo da pasta iteracoesXpeso}

O algoritmo memético apresentou um desempenho semelhante, atingindo soluções próximas do ótimo em todas as execuções com pesos superiores a 100. Além disso, sua convergência foi extremamente rápida, sendo que em algumas execuções uma única iteração foi suficiente para encontrar uma solução próxima do ótimo, com um valor aproximadamente uma unidade acima do ideal. Embora o gráfico do mínimo global pudesse sugerir uma estagnação, através do o gráfico do mínimo local é possível ver o funcionamento do algoritmo na busca por soluções melhores.
\todo{colocar o conteúdo da pasta convergenciaMA}

Por outro lado, ao utilizar pesos de penalização mais baixos (50 ou inferiores), ambos os algoritmos começaram a gerar soluções inviáveis. À medida que o peso diminuía, as soluções aparentavam melhorar em termos de qualidade, mas violavam cada vez mais as restrições, resultando em soluções estruturalmente inviáveis.

\todo{colocar as imagens da pasta solucoesInviaveis}


% \begin{table}[!ht]
%     \resizebox{\textwidth}{!}{%
%         \begin{tabular}{llllrrrr}
%             \bottomrule
%             \textbf{Obj.\ func.}                       &
%             \textbf{Crossover}                         &
%             \textbf{Selection}                         &
%             \textbf{\( \boldsymbol{K} \)}              &
%             \textbf{Best fitness}                      &
%             \textbf{Average fitness}                   &
%             \textbf{Fitness \( \boldsymbol{\sigma} \)} &
%             \textbf{Discrepancy}                                                                                                                           \\ \midrule
%             F12014                                     & one\_point & roulette   & None & 31,333,596.32  & 51,840,884.76  & 13,297,083.01 & 51,840,784.76  \\
%             F12014                                     & one\_point & tournament & 0.1  & 6,459,590.39   & 8,170,932.28   & 1,340,563.86  & 8,170,832.28   \\
%             F12014                                     & one\_point & tournament & 0.2  & 501,184.77     & 1,011,075.90   & 513,203.52    & 1,010,975.90   \\
%             F12014                                     & one\_point & tournament & 0.3  & 216,342.15     & 703,931.58     & 525,414.34    & 703,831.58     \\
%             F12014                                     & one\_point & tournament & 0.4  & 213,078.25     & 699,594.39     & 317,577.00    & 699,494.39     \\
%             F12014                                     & one\_point & tournament & 0.5  & 257,286.41     & 850,689.20     & 462,953.51    & 850,589.20     \\
%             F12014                                     & arithmetic & roulette   & None & 149,508,377.65 & 194,305,830.34 & 31,394,179.76 & 194,305,730.34 \\
%             F12014                                     & arithmetic & tournament & 0.1  & 4,554,550.01   & 6,522,961.69   & 1,494,523.52  & 6,522,861.69   \\
%             F12014                                     & arithmetic & tournament & 0.2  & 418,897.70     & 658,253.99     & 124,383.49    & 658,153.99     \\
%             F12014                                     & arithmetic & tournament & 0.3  & 202,876.41     & 402,340.97     & 176,709.61    & 402,240.97     \\
%             F12014                                     & arithmetic & tournament & 0.4  & 143,574.03     & 506,514.55     & 323,014.25    & 506,414.55     \\
%             F12014                                     & arithmetic & tournament & 0.5  & 124,955.96     & 409,121.44     & 215,047.98    & 409,021.44     \\
%             F52014                                     & one\_point & roulette   & None & 520.19         & 520.24         & 0.03          & 20.24          \\
%             F52014                                     & one\_point & tournament & 0.1  & 520.16         & 520.18         & 0.02          & 20.18          \\
%             F52014                                     & one\_point & tournament & 0.2  & 520.04         & 520.07         & 0.02          & 20.07          \\
%             F52014                                     & one\_point & tournament & 0.3  & 520.06         & 520.06         & 0.01          & 20.06          \\
%             F52014                                     & one\_point & tournament & 0.4  & 520.04         & 520.05         & 0.01          & 20.05          \\
%             F52014                                     & one\_point & tournament & 0.5  & 520.04         & 520.05         & 0.01          & 20.05          \\
%             F52014                                     & arithmetic & roulette   & None & 520.60         & 520.70         & 0.07          & 20.70          \\
%             F52014                                     & arithmetic & tournament & 0.1  & 520.53         & 520.70         & 0.07          & 20.70          \\
%             F52014                                     & arithmetic & tournament & 0.2  & 509.85         & 518.68         & 4.02          & 18.68          \\
%             F52014                                     & arithmetic & tournament & 0.3  & 509.64         & 517.27         & 5.14          & 17.27          \\
%             F52014                                     & arithmetic & tournament & 0.4  & 520.30         & 520.38         & 0.03          & 20.38          \\
%             F52014                                     & arithmetic & tournament & 0.5  & 520.26         & 520.31         & 0.03          & 20.31          \\
%             \toprule
%         \end{tabular}%
%     }
%     \caption{Resultados obtidos para os problemas \glsentryname{f1} e \glsentryname{f5} em 20 dimensões.
%     }%
%     \label{tab:fitness_20}
% \end{table}



% Nos casos em que o operador de seleção por torneio foi utilizado para a função \gls{f1}, o valor de \( K \) foi modificado entre 0,1 e 0,5 aumentando sempre em 0,1.
% Neste caso foi possível observar uma ligeira piora da solução em ambos os extremos.
% O valor que encontrou os melhores resultados foi 0,2 (padrão da biblioteca).
% Foi bastante notório que este valor levava o algoritmo a convergir mais tarde, com cerca de 4000 gerações.
% Os valores associados ao elitismo não foram alterados, mantendo-se os padrões da biblioteca, com 0,1 para o elitismo de melhores indivíduos e 0,3 para o de piores.

% Acerca das funções de benchmark, a \gls{f1} apresentou resultados menos satisfatórios, com discrepâncias de até 15.026.094,68 e 194.305.730,34 em 10 e 20 dimensões, respectivamente.
% Estes valores são bastante elevados, considerando que a função \gls{f1} é convexa e possui um mínimo global conhecido de 100.

% Por outro lado, a função \gls{f5} apresentou resultados mais satisfatórios, com discrepâncias de até 20,23 e 20,70 em 10 e 20 dimensões, respectivamente, levando em consideração que a função \gls{f5} é não convexa e possui um mínimo global conhecido de 500.
% Estes valores são mais próximos do esperado, indicando que o \gls{ga} apresenta características que o tornam mais eficiente para este tipo de função.

% A seleção por roleta esteve entre os piores resultados. Seus valores de \gls{fitness} médio figuravam como 15.026.194,68 e 194.305.830,34, para 10 e 20 dimensões.
% A seleção por torneio, por outro lado, apresentou resultados mais satisfatórios.
% Considerando os valores de \gls{fitness} médios, a seleção por torneio com \( K \) em 0.3 foi a que obteve os melhores resultados para a função \gls{f1} em 20 dimensões, com 402,340.97.
% Já para o experimento com 10 dimensões, houve uma melhora contínua conforme se aumentava o \( K \), até 0.5, com 143.231,77.
% É possível propor a hipótese de que uma menor pressão seletiva, como a proporcionada por \( K \) em valores maiores, pode levar a uma maior diversidade genética, o que pode ser benéfico para a convergência do algoritmo.

% Apesar de o operador de \gls{crossover} por aritmética ter apresentado o pior resultado para a função \gls{f1}, com 15.026.194,68 para 10 dimensões, deve-se observar que, neste caso, ele estava associado à seleção por roleta.
% Quando o observa junto à seleção por torneio com \( K \) em 0.5, o valor de adequação médio obtido foi de 143.231,77, o melhor para essa função nesse número de dimensões.

% A função \gls{f5}, por apresentar menos diferença entre os valores de adequação das soluções das variantes, tornou mais difícil a identificação de padrões.
% Ainda assim, pôde-se perceber que em 10 dimensões, com o torneio exibindo \( K \) em 0.4 e 0.3 associado ao operador de \gls{crossover} um ponto teve vantagem em relação aos outros valores de \( k \) e roleta.
% De todo modo, o operador de \gls{crossover} aritmético obtém vantagem significativa com qualquer valor de \( K \) maior que 0.1.

% Quanto à análise com 20 dimensões, a diferença entre as variantes foi ainda menos acentuada.
% Apenas se constatou que o operador de \gls{crossover} aritmético gerou resultados ligeiramente superiores quando o torneio foi utilizado com \( K \) em 0.2 e 0.3.